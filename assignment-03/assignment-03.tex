\documentclass[]{article}
\usepackage{lmodern}
\usepackage{amssymb,amsmath}
\usepackage{ifxetex,ifluatex}
\usepackage{fixltx2e} % provides \textsubscript
\ifnum 0\ifxetex 1\fi\ifluatex 1\fi=0 % if pdftex
  \usepackage[T1]{fontenc}
  \usepackage[utf8]{inputenc}
\else % if luatex or xelatex
  \ifxetex
    \usepackage{mathspec}
  \else
    \usepackage{fontspec}
  \fi
  \defaultfontfeatures{Ligatures=TeX,Scale=MatchLowercase}
\fi
% use upquote if available, for straight quotes in verbatim environments
\IfFileExists{upquote.sty}{\usepackage{upquote}}{}
% use microtype if available
\IfFileExists{microtype.sty}{%
\usepackage{microtype}
\UseMicrotypeSet[protrusion]{basicmath} % disable protrusion for tt fonts
}{}
\usepackage[unicode=true]{hyperref}
\hypersetup{
            pdftitle={Assignment 3: Iterators, Function Templates},
            pdfborder={0 0 0},
            breaklinks=true}
\urlstyle{same}  % don't use monospace font for urls
\usepackage{color}
\usepackage{fancyvrb}
\newcommand{\VerbBar}{|}
\newcommand{\VERB}{\Verb[commandchars=\\\{\}]}
\DefineVerbatimEnvironment{Highlighting}{Verbatim}{commandchars=\\\{\}}
% Add ',fontsize=\small' for more characters per line
\newenvironment{Shaded}{}{}
\newcommand{\KeywordTok}[1]{\textcolor[rgb]{0.00,0.44,0.13}{\textbf{{#1}}}}
\newcommand{\DataTypeTok}[1]{\textcolor[rgb]{0.56,0.13,0.00}{{#1}}}
\newcommand{\DecValTok}[1]{\textcolor[rgb]{0.25,0.63,0.44}{{#1}}}
\newcommand{\BaseNTok}[1]{\textcolor[rgb]{0.25,0.63,0.44}{{#1}}}
\newcommand{\FloatTok}[1]{\textcolor[rgb]{0.25,0.63,0.44}{{#1}}}
\newcommand{\ConstantTok}[1]{\textcolor[rgb]{0.53,0.00,0.00}{{#1}}}
\newcommand{\CharTok}[1]{\textcolor[rgb]{0.25,0.44,0.63}{{#1}}}
\newcommand{\SpecialCharTok}[1]{\textcolor[rgb]{0.25,0.44,0.63}{{#1}}}
\newcommand{\StringTok}[1]{\textcolor[rgb]{0.25,0.44,0.63}{{#1}}}
\newcommand{\VerbatimStringTok}[1]{\textcolor[rgb]{0.25,0.44,0.63}{{#1}}}
\newcommand{\SpecialStringTok}[1]{\textcolor[rgb]{0.73,0.40,0.53}{{#1}}}
\newcommand{\ImportTok}[1]{{#1}}
\newcommand{\CommentTok}[1]{\textcolor[rgb]{0.38,0.63,0.69}{\textit{{#1}}}}
\newcommand{\DocumentationTok}[1]{\textcolor[rgb]{0.73,0.13,0.13}{\textit{{#1}}}}
\newcommand{\AnnotationTok}[1]{\textcolor[rgb]{0.38,0.63,0.69}{\textbf{\textit{{#1}}}}}
\newcommand{\CommentVarTok}[1]{\textcolor[rgb]{0.38,0.63,0.69}{\textbf{\textit{{#1}}}}}
\newcommand{\OtherTok}[1]{\textcolor[rgb]{0.00,0.44,0.13}{{#1}}}
\newcommand{\FunctionTok}[1]{\textcolor[rgb]{0.02,0.16,0.49}{{#1}}}
\newcommand{\VariableTok}[1]{\textcolor[rgb]{0.10,0.09,0.49}{{#1}}}
\newcommand{\ControlFlowTok}[1]{\textcolor[rgb]{0.00,0.44,0.13}{\textbf{{#1}}}}
\newcommand{\OperatorTok}[1]{\textcolor[rgb]{0.40,0.40,0.40}{{#1}}}
\newcommand{\BuiltInTok}[1]{{#1}}
\newcommand{\ExtensionTok}[1]{{#1}}
\newcommand{\PreprocessorTok}[1]{\textcolor[rgb]{0.74,0.48,0.00}{{#1}}}
\newcommand{\AttributeTok}[1]{\textcolor[rgb]{0.49,0.56,0.16}{{#1}}}
\newcommand{\RegionMarkerTok}[1]{{#1}}
\newcommand{\InformationTok}[1]{\textcolor[rgb]{0.38,0.63,0.69}{\textbf{\textit{{#1}}}}}
\newcommand{\WarningTok}[1]{\textcolor[rgb]{0.38,0.63,0.69}{\textbf{\textit{{#1}}}}}
\newcommand{\AlertTok}[1]{\textcolor[rgb]{1.00,0.00,0.00}{\textbf{{#1}}}}
\newcommand{\ErrorTok}[1]{\textcolor[rgb]{1.00,0.00,0.00}{\textbf{{#1}}}}
\newcommand{\NormalTok}[1]{{#1}}
\IfFileExists{parskip.sty}{%
\usepackage{parskip}
}{% else
\setlength{\parindent}{0pt}
\setlength{\parskip}{6pt plus 2pt minus 1pt}
}
\setlength{\emergencystretch}{3em}  % prevent overfull lines
\providecommand{\tightlist}{%
  \setlength{\itemsep}{0pt}\setlength{\parskip}{0pt}}
\setcounter{secnumdepth}{0}
% Redefines (sub)paragraphs to behave more like sections
\ifx\paragraph\undefined\else
\let\oldparagraph\paragraph
\renewcommand{\paragraph}[1]{\oldparagraph{#1}\mbox{}}
\fi
\ifx\subparagraph\undefined\else
\let\oldsubparagraph\subparagraph
\renewcommand{\subparagraph}[1]{\oldsubparagraph{#1}\mbox{}}
\fi

\title{Assignment 3: Iterators, Function Templates}
\date{}

\begin{document}
\maketitle

\textbf{C++ Programming Course, Winter Term 2016}

\section{3-1: Algorithms / Function
Templates}\label{algorithms-function-templates}

Your implementations should be a combination of functions provided by
the STL and must not contain explicit loops like \texttt{for} or
\texttt{while}.

You may use any algorithm interface defined in the C++11 standard. For
example, you may use \texttt{std::minmax} (C++11) in your
implementations, but not \texttt{std::for\_each\_n} (C++17).

References:

\begin{itemize}
\tightlist
\item
  Talk by Marshall Clow: ``STL Algorithms - why you should use them, and
  how to write your own''
  \url{https://www.youtube.com/watch?v=h4Jl1fk3MkQ}
\item
  C++ Algorithm Library:
  \url{http://en.cppreference.com/w/cpp/algorithm}
\item
  C++ Iterator Library:
  \url{http://en.cppreference.com/w/cpp/header/iterator}
\end{itemize}

\subsection{3-1-1: find\_mean\_rep}\label{find_mean_rep}

Implement a function template \texttt{cpppc::find\_mean\_rep} which
accepts a range specified by two iterators in the \texttt{InputIterator}
category and returns an iterator to the element in the range that is
closest to the mean of the range.

Your implementation should be automatically more efficient for random
access iterator ranges without specialization.

\textbf{Function interface:}

\begin{Shaded}
\begin{Highlighting}[]
\KeywordTok{template} \NormalTok{<}\KeywordTok{typename} \NormalTok{InputIter>}
\NormalTok{InputIter find_mean_rep(InputIter first, InputIter last);}
\end{Highlighting}
\end{Shaded}

\textbf{Example:}

\begin{Shaded}
\begin{Highlighting}[]
\BuiltInTok{std::}\NormalTok{vector<}\DataTypeTok{int}\NormalTok{> v \{ }\DecValTok{1}\NormalTok{, }\DecValTok{2}\NormalTok{, }\DecValTok{3}\NormalTok{, }\DecValTok{30}\NormalTok{, }\DecValTok{40}\NormalTok{, }\DecValTok{50} \NormalTok{\}; }\CommentTok{// mean: 21}
\KeywordTok{auto} \NormalTok{closest_to_mean_it = cpppc::find_mean_rep(v.begin(), v.end());}
\CommentTok{// -> iterator at index 3 (|21-30| = 9)}
\end{Highlighting}
\end{Shaded}

\subsection{3-1-2: histogram
(RandomAccessIterator)}\label{histogram-randomaccessiterator}

Implement a function template \texttt{cpppc::histogram} which accepts a
range specified by two iterators in the \texttt{InputIterator} category
and replaces each value by the number of its occurrences in the range.
Values are unique in the result histogram and ordered by their first
occurrence in the range. The function returns an iterator past the final
element in the histogram.

Only integer values (int, long, size\_t, \ldots{}) have to be supported.

\textbf{Function interface:}

\begin{Shaded}
\begin{Highlighting}[]
\KeywordTok{template} \NormalTok{<}\KeywordTok{typename} \NormalTok{RAIt>}
\NormalTok{RAIt histogram(RAIt first, RAIt last);}
\end{Highlighting}
\end{Shaded}

\textbf{Example:}

\begin{Shaded}
\begin{Highlighting}[]
\BuiltInTok{std::}\NormalTok{vector<}\DataTypeTok{int}\NormalTok{> v \{ }\DecValTok{1}\NormalTok{, }\DecValTok{5}\NormalTok{, }\DecValTok{5}\NormalTok{, }\DecValTok{3}\NormalTok{, }\DecValTok{4}\NormalTok{, }\DecValTok{1}\NormalTok{, }\DecValTok{5}\NormalTok{, }\DecValTok{7} \NormalTok{\};}
\KeywordTok{auto} \NormalTok{hist_end = cpppc::histogram(v.begin(), v.end());}
\CommentTok{// -> 1: 2 occurences}
\CommentTok{//    5: 3 occurences}
\CommentTok{//    5: skipped}
\CommentTok{//    ...}
\CommentTok{// -> \{ 2, 3, 1, 1, 1 \}  <- occurrences}
\CommentTok{//      ^  ^  ^  ^  ^}
\CommentTok{//      |  |  |  |  |}
\CommentTok{//      1  5  3  4  7    <- value}
\CommentTok{//}
\CommentTok{// -> returns iterator at position 5}
\end{Highlighting}
\end{Shaded}

\section{3-2: Custom Container /
Iterator}\label{custom-container-iterator}

Implement a sparse array (see
\url{https://en.wikipedia.org/wiki/Sparse_array}) container that
satisfies the \texttt{std::array} interface.

A sparse array is a usually large array with most elements set to a
default value like 0. Instead of allocating memory for all values, only
non-default values are actually allocated.

Different from vectors, arrays are static containers so their size does
not change after instantiation. The values of its elements can be
iterated and changed, but elements cannot be added or removed.

References:

\begin{itemize}
\tightlist
\item
  std::array in the C++ Standard Template Library:
  \url{http://en.cppreference.com/w/cpp/container/array}
\end{itemize}

\begin{Shaded}
\begin{Highlighting}[]
\DataTypeTok{int} \ControlFlowTok{default} \NormalTok{= }\DecValTok{0}\NormalTok{;}
\NormalTok{cpppc::sparse_array<}\DataTypeTok{int}\NormalTok{, }\DecValTok{10000}\NormalTok{> sa(}\ControlFlowTok{default}\NormalTok{);}
\CommentTok{// no actual values in `sa` yet, all set to default}
\DataTypeTok{size_t} \NormalTok{sa_size = sa.size(); }\CommentTok{// -> 10000}
\NormalTok{assert(sa[}\DecValTok{345}\NormalTok{] == }\DecValTok{0}\NormalTok{);       }\CommentTok{// returns `default` if no value}
                            \CommentTok{// set at position}

\NormalTok{sa[}\DecValTok{230}\NormalTok{] = }\DecValTok{23}\NormalTok{;}
\NormalTok{assert(sa[}\DecValTok{230}\NormalTok{] == }\DecValTok{23}\NormalTok{);}

\NormalTok{sa[}\DecValTok{420}\NormalTok{] = }\DecValTok{42}\NormalTok{;}
\CommentTok{// two values stored in `sa`}

\CommentTok{// Must be compatible with STL algorithms:}

\KeywordTok{auto} \NormalTok{found_42 = }\BuiltInTok{std::}\NormalTok{find(sa.begin(), sa.end(), }\DecValTok{42}\NormalTok{);}
\CommentTok{// -> iterator at 'virtual' position 420 (the second 'real' value)}
\end{Highlighting}
\end{Shaded}

\textbf{Notes:}

\begin{itemize}
\tightlist
\item
  Your implementation has to detect assignment of a value reference. Try
  a temporary proxy:
  \url{https://en.wikibooks.org/wiki/More_C++_Idioms/Temporary_Proxy}.
\item
  Obviously, the \texttt{sparse\_array} cannot provide direct access to
  its underlying array (as it doesn't exist) so method \texttt{data()}
  is not provided.
\end{itemize}

\section{3-X: Challenges}\label{x-challenges}

\subsection{3-X-1: Skip List}\label{x-1-skip-list}

Implement a class template \texttt{cpppc::map} that satisfies the
\texttt{std::map} container interface based on a Skip List data
structure.

References:

\begin{itemize}
\tightlist
\item
  Specification of std::map:
  \url{http://en.cppreference.com/w/cpp/container/map}
\item
  \emph{Skip Lists: A Probabilistic Alternative to Balanced Trees}. W.
  Pugh, 1990. \url{http://cpppc.pub.lab.nm.ifi.lmu.de/docs/skiplist.pdf}
\end{itemize}

\end{document}

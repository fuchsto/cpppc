\documentclass[]{article}
\usepackage{lmodern}
\usepackage{amssymb,amsmath}
\usepackage{ifxetex,ifluatex}
\usepackage{fixltx2e} % provides \textsubscript
\ifnum 0\ifxetex 1\fi\ifluatex 1\fi=0 % if pdftex
  \usepackage[T1]{fontenc}
  \usepackage[utf8]{inputenc}
\else % if luatex or xelatex
  \ifxetex
    \usepackage{mathspec}
  \else
    \usepackage{fontspec}
  \fi
  \defaultfontfeatures{Mapping=tex-text,Scale=MatchLowercase}
  \newcommand{\euro}{€}
\fi
% use upquote if available, for straight quotes in verbatim environments
\IfFileExists{upquote.sty}{\usepackage{upquote}}{}
% use microtype if available
\IfFileExists{microtype.sty}{%
\usepackage{microtype}
\UseMicrotypeSet[protrusion]{basicmath} % disable protrusion for tt fonts
}{}
\makeatletter
\@ifpackageloaded{hyperref}{}{%
\ifxetex
  \usepackage[setpagesize=false, % page size defined by xetex
              unicode=false, % unicode breaks when used with xetex
              xetex]{hyperref}
\else
  \usepackage[unicode=true]{hyperref}
\fi
}
\@ifpackageloaded{color}{
    \PassOptionsToPackage{usenames,dvipsnames}{color}
}{%
    \usepackage[usenames,dvipsnames]{color}
}
\makeatother
\hypersetup{breaklinks=true,
            bookmarks=true,
            pdfauthor={},
            pdftitle={Assignment 2: Introduction to Templates},
            colorlinks=true,
            citecolor=blue,
            urlcolor=blue,
            linkcolor=magenta,
            pdfborder={0 0 0}
            }
\urlstyle{same}  % don't use monospace font for urls
\usepackage{color}
\usepackage{fancyvrb}
\newcommand{\VerbBar}{|}
\newcommand{\VERB}{\Verb[commandchars=\\\{\}]}
\DefineVerbatimEnvironment{Highlighting}{Verbatim}{commandchars=\\\{\}}
% Add ',fontsize=\small' for more characters per line
\newenvironment{Shaded}{}{}
\newcommand{\KeywordTok}[1]{\textcolor[rgb]{0.00,0.44,0.13}{\textbf{{#1}}}}
\newcommand{\DataTypeTok}[1]{\textcolor[rgb]{0.56,0.13,0.00}{{#1}}}
\newcommand{\DecValTok}[1]{\textcolor[rgb]{0.25,0.63,0.44}{{#1}}}
\newcommand{\BaseNTok}[1]{\textcolor[rgb]{0.25,0.63,0.44}{{#1}}}
\newcommand{\FloatTok}[1]{\textcolor[rgb]{0.25,0.63,0.44}{{#1}}}
\newcommand{\ConstantTok}[1]{\textcolor[rgb]{0.53,0.00,0.00}{{#1}}}
\newcommand{\CharTok}[1]{\textcolor[rgb]{0.25,0.44,0.63}{{#1}}}
\newcommand{\SpecialCharTok}[1]{\textcolor[rgb]{0.25,0.44,0.63}{{#1}}}
\newcommand{\StringTok}[1]{\textcolor[rgb]{0.25,0.44,0.63}{{#1}}}
\newcommand{\VerbatimStringTok}[1]{\textcolor[rgb]{0.25,0.44,0.63}{{#1}}}
\newcommand{\SpecialStringTok}[1]{\textcolor[rgb]{0.73,0.40,0.53}{{#1}}}
\newcommand{\ImportTok}[1]{{#1}}
\newcommand{\CommentTok}[1]{\textcolor[rgb]{0.38,0.63,0.69}{\textit{{#1}}}}
\newcommand{\DocumentationTok}[1]{\textcolor[rgb]{0.73,0.13,0.13}{\textit{{#1}}}}
\newcommand{\AnnotationTok}[1]{\textcolor[rgb]{0.38,0.63,0.69}{\textbf{\textit{{#1}}}}}
\newcommand{\CommentVarTok}[1]{\textcolor[rgb]{0.38,0.63,0.69}{\textbf{\textit{{#1}}}}}
\newcommand{\OtherTok}[1]{\textcolor[rgb]{0.00,0.44,0.13}{{#1}}}
\newcommand{\FunctionTok}[1]{\textcolor[rgb]{0.02,0.16,0.49}{{#1}}}
\newcommand{\VariableTok}[1]{\textcolor[rgb]{0.10,0.09,0.49}{{#1}}}
\newcommand{\ControlFlowTok}[1]{\textcolor[rgb]{0.00,0.44,0.13}{\textbf{{#1}}}}
\newcommand{\OperatorTok}[1]{\textcolor[rgb]{0.40,0.40,0.40}{{#1}}}
\newcommand{\BuiltInTok}[1]{{#1}}
\newcommand{\ExtensionTok}[1]{{#1}}
\newcommand{\PreprocessorTok}[1]{\textcolor[rgb]{0.74,0.48,0.00}{{#1}}}
\newcommand{\AttributeTok}[1]{\textcolor[rgb]{0.49,0.56,0.16}{{#1}}}
\newcommand{\RegionMarkerTok}[1]{{#1}}
\newcommand{\InformationTok}[1]{\textcolor[rgb]{0.38,0.63,0.69}{\textbf{\textit{{#1}}}}}
\newcommand{\WarningTok}[1]{\textcolor[rgb]{0.38,0.63,0.69}{\textbf{\textit{{#1}}}}}
\newcommand{\AlertTok}[1]{\textcolor[rgb]{1.00,0.00,0.00}{\textbf{{#1}}}}
\newcommand{\ErrorTok}[1]{\textcolor[rgb]{1.00,0.00,0.00}{\textbf{{#1}}}}
\newcommand{\NormalTok}[1]{{#1}}
\usepackage{longtable,booktabs}
\setlength{\parindent}{0pt}
\setlength{\parskip}{6pt plus 2pt minus 1pt}
\setlength{\emergencystretch}{3em}  % prevent overfull lines
\providecommand{\tightlist}{%
  \setlength{\itemsep}{0pt}\setlength{\parskip}{0pt}}
\setcounter{secnumdepth}{0}

\title{Assignment 2: Introduction to Templates}
\date{}

% Redefines (sub)paragraphs to behave more like sections
\ifx\paragraph\undefined\else
\let\oldparagraph\paragraph
\renewcommand{\paragraph}[1]{\oldparagraph{#1}\mbox{}}
\fi
\ifx\subparagraph\undefined\else
\let\oldsubparagraph\subparagraph
\renewcommand{\subparagraph}[1]{\oldsubparagraph{#1}\mbox{}}
\fi

\begin{document}
\maketitle

\textbf{C++ Programming Course, Winter Term 2016}

\section{2-0: Prerequisites}\label{prerequisites}

\subsection{2-0-1: Iterator Concepts}\label{iterator-concepts}

\begin{itemize}
\tightlist
\item
  Make yourself familiar with the \texttt{Iterator} concepts in the
  STL:\\
   \url{http://en.cppreference.com/w/cpp/concept/Iterator}
\item
  What are the differences between the concepts \texttt{ForwardIterator}
  and \texttt{RandomAccessIterator}?
\item
  What are the differences between the concepts \texttt{InputIterator}
  and \texttt{OutputIterator}?
\end{itemize}

\subsection{2-0-2: Sequence Container
Concept}\label{sequence-container-concept}

Sequence containers implement data structures which can be accessed
sequentially. Methods \texttt{begin()} and \texttt{end()} define the
iteration space of the container elements.

\begin{itemize}
\tightlist
\item
  Make yourself familiar with \emph{Sequence Container} concept defined
  in the STL:\\
   \url{http://en.cppreference.com/w/cpp/concept/SequenceContainer}
\end{itemize}

\textbf{Excerpt}:

\begin{longtable}[c]{@{}ll@{}}
\toprule
Type & Synopsis\tabularnewline
\midrule
\endhead
\texttt{typename\ \ \ \ \ \ value\_type} & the container's element type
\texttt{T}\tabularnewline
\texttt{typename\ \ \ \ \ \ \ \ iterator} & iterator type referencing a
container element\tabularnewline
\texttt{typename\ \ const\_iterator} & typically defined as
\texttt{const\ iterator}\tabularnewline
\texttt{typename\ \ \ \ \ \ \ reference} & type definition for
\texttt{value\_type\ \&}\tabularnewline
\texttt{typename\ const\_reference} & type definition for
\texttt{const\ value\_type\ \&}\tabularnewline
\bottomrule
\end{longtable}

\begin{longtable}[c]{@{}ll@{}}
\toprule
Signature & Synopsis\tabularnewline
\midrule
\endhead
\texttt{iterator\ begin()\ \ \ \ \ \ \ \ \ \ \ \ } &
\vtop{\hbox{\strut iterator referencing the first element in
the}\hbox{\strut  container or \texttt{end()} if container is
empty}}\tabularnewline
\texttt{const\_iterator\ begin()\ \ const} & \vtop{\hbox{\strut const
iterator referencing the first element in the}\hbox{\strut  container or
\texttt{end()} if container is empty}}\tabularnewline
\texttt{iterator\ end()\ \ \ \ \ \ \ \ \ \ \ \ \ \ } &
\vtop{\hbox{\strut iterator referencing past the final element
in}\hbox{\strut  the container}}\tabularnewline
\texttt{const\_iterator\ end()\ \ \ \ const} & \vtop{\hbox{\strut const
iterator referencing past the final element in}\hbox{\strut  the
container}}\tabularnewline
\texttt{size\_type\ size()\ \ \ \ \ \ \ \ const} &
\vtop{\hbox{\strut number of elements in the container, same
as}\hbox{\strut  \texttt{end()\ -\ begin()}}}\tabularnewline
\bottomrule
\end{longtable}

\section{2-1: The Measurements\textless{}T\textgreater{} Class
Template}\label{the-measurementst-class-template}

Assuming you run a series of benchmarks, each returning a measurement.
At the end of the test series, the mean, median, standard deviation
(sigma) and variance should be printed.

Implement the class template
\texttt{Measurements\textless{}T\textgreater{}} representing a sequence
container that allows to collect measurement data as single values and
provides methods to obtain the mean, median, standard deviation and
variance of the container elements.

\subsection{Measurements Container
Concept}\label{measurements-container-concept}

In addition to the Sequence Container Concept:

\begin{longtable}[c]{@{}ll@{}}
\toprule
Signature & Synopsis\tabularnewline
\midrule
\endhead
\texttt{T\ median()\ \ \ \ \ \ \ \ \ \ \ \ \ const} &
\vtop{\hbox{\strut returns the median of the elements in the
container}\hbox{\strut  or \texttt{0} if the container is
empty.}}\tabularnewline
\texttt{double\ mean()\ \ \ \ \ \ \ \ \ \ const} & returns the mean of
the elements in the container\tabularnewline
\texttt{double\ variance()\ \ \ \ \ \ const} &
\vtop{\hbox{\strut returns the population variance of the elements
in}\hbox{\strut  the container}}\tabularnewline
\texttt{double\ sigma()\ \ \ \ \ \ \ \ \ const} &
\vtop{\hbox{\strut returns the standard deviation of the elements
in}\hbox{\strut  the container}}\tabularnewline
\bottomrule
\end{longtable}

\subsection{Example:}\label{example}

\begin{Shaded}
\begin{Highlighting}[]
\NormalTok{Measurements<}\DataTypeTok{int}\NormalTok{> m1;}
\NormalTok{m1.insert(}\DecValTok{10}\NormalTok{);}
\NormalTok{m1.insert(}\DecValTok{34}\NormalTok{);}
\NormalTok{m1.size(); }\CommentTok{// = 2}
\NormalTok{Measurements<}\DataTypeTok{double}\NormalTok{> m2;}
\NormalTok{std::vector<}\DataTypeTok{double}\NormalTok{> v(\{ }\DecValTok{36}\NormalTok{, }\DecValTok{37}\NormalTok{, }\DecValTok{10} \NormalTok{\});}
\NormalTok{m2.insert(v.begin(), v.end());}

\NormalTok{m1.insert(m2.begin(), m2.end())}

\NormalTok{m1.size(); }\CommentTok{// = 5}

\DataTypeTok{int}    \NormalTok{median = m1.median();}
\DataTypeTok{double} \NormalTok{mean   = m1.mean();}
\DataTypeTok{double} \NormalTok{sdev   = m1.sigma();}
\DataTypeTok{double} \NormalTok{var    = m1.variance();}
\end{Highlighting}
\end{Shaded}

Define a class template \texttt{Measurements\textless{}T\textgreater{}}
that satisfies the Sequence Container concept
(\url{http://en.cppreference.com/w/cpp/concept/SequenceContainer}) and
the Measurements Container concept defined above.

You may ignore the \texttt{emplace} methods for now.

The solution uses \texttt{std::vector} as a starting point, but you may
use any underlying data structure in your implementation of
\texttt{cpppc::Measurements\textless{}T\textgreater{}}.

\section{2-X: Improve Efficiency}\label{x-improve-efficiency}

\begin{itemize}
\tightlist
\item
  Refactor your implementation of
  \texttt{Measurements\textless{}T\textgreater{}} such that all methods
  in the \emph{Measurements} concept maintain constant computational
  complexity \(O(c)\)
\item
  There are arithmetic solutions, possibly at the cost of numeric
  stability, and approaches focusing on the underlying data structure
\end{itemize}

\subsection{Hints}\label{hints}

\begin{itemize}
\tightlist
\item
  Just for fun, have a look at
  \url{https://en.wikipedia.org/wiki/Standard_deviation\#Rapid_calculation_methods}
\item
  Review trustworthy (!) references for \textbf{multi-index containers}
\item
  DrDobbs is a very trustworthy reference
\end{itemize}

\end{document}
